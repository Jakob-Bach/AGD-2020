\documentclass[headinclude,headsepline]{scrartcl}

%packages
\usepackage[ngerman]{babel} % hyphenation
\usepackage{scrlayer-scrpage} % header
\usepackage{hyperref} % links
\usepackage{paralist} % enumerations
\usepackage[margin=1.15in]{geometry}
\usepackage{enumitem} % enumerations

%commands
\ihead{
	\textbf{Feedback geben}
}
\ohead{
	Analysetechniken für große Datenbestände WS 20/21
}
\hypersetup{
	colorlinks=true, % false: boxed links; true: colored links
	linkcolor=black, % color of internal links (change box color with linkbordercolor)
	citecolor=black, % color of links to bibliography
	filecolor=black, % color of file links
	urlcolor=blue % color of external links
}

\begin{document}

\section*{Ablauf}

Falls Sie anonymes Peer-Feedback zum Code geben und erhalten wollen, laden Sie Ihre Lösung zu den Programmieraufgaben innerhalb der Frist in das ILIAS-Übungsobjekt ``Abgaben'' hoch.
Laden Sie bitte keine Lösung hoch, wenn Sie kein Feedback geben wollen, weil sonst Ihre zugelosten Feedbackpartner:innen erst einmal mit leeren Händen dastehen.
Feedback kann erst nach Ablauf der Abgabefrist gegeben werden.
ILIAS zeigt Ihnen die Frist für das Geben von Feedback somit nach Verstreichen der Abgabefrist an.
Nach Ablauf der Feedbackfrist wird dann das erhaltene Feedback sichtbar.
Sollten die Ihnen zugeteilten Feedback-Partner:innen kein Feedback oder kein brauchbares Feedback gegeben haben,
können Sie Feedback vom Übungsleiter erhalten, indem Sie eine Mail an \url{mailto:jakob.bach@kit.edu} schreiben.

\section*{Inhalt}

In der Form des Feedbacks sind Sie frei.
Sie können gerne Stichpunkte verwenden.
Achten Sie darauf, dass ihr Feedback eine gewisse Länge hat.
Für Textfeedback ist eine Mindestlänge von 500 Zeichen eingestellt.
Sie können auch eine Feedback-Datei hochladen (wenn sie z.B. ihr Feedback formatieren wollen).

Grundidee des Feedbacks ist, dass Sie vom Code anderer lernen können und wiederum Ihrerseits Kommiliton:innen Verbesserungsvorschläge machen.
Geben Sie sowohl Feedback zur Lösung insgesamt als auch zu einzelnen Teilaufgaben / Code-Abschnitten, bei denen Ihnen etwas Besonderes auffällt.
Sie können sich an folgenden Fragen orientieren (diese Fragen müssen aber nicht Schritt für Schritt abgearbeitet werden):

\vspace{10pt}

\begin{compactenum}[a)]
	\item Was ist Ihr Gesamteindruck von der Lösung?
	\item Ist die Lösung korrekt?
	\item Wie beurteilen Sie die Code-Qualität?
	\item Überzeugt Sie die Interpretation der Ergebnisse (wenn vorhanden)?
	\item Wie verständlich sind Code und Interpretation der Ergebnisse?
	\item Lassen sich bestimmte Teile des Codes einfacher, eleganter oder effizienter schreiben?
	\item Was konnten Sie aus der Lösung lernen?
\end{compactenum}

\vspace{10pt}

Achten Sie darauf, konstruktiv zu bleiben und auch positive Punkte zu erwähnen.

\end{document}
